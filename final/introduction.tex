\section{Introduction}
\label{intro}

Content sharing has been the key ingredient of the biggest success stories in the last decade. 
Social networks like Facebook and LinkedIn, video sharing services like YouTube, 
Ecommerce websites like Amazon and online travel agencies like TripAdviser and Expedia are all examples
of online platforms that provide a convenient way for their users to exchange content, goods or services.
It is obvious that the success of such online platforms depends on how effectively they link the end users to relevant content.
This latter cannot be achieved without proper content classification and user feedback analysis.
This motivated us to apply machine learning techniques to automate the process of user feedback analysis and content classification.
We work with a real sample dataset taken from YouTube~\cite{youtubedata}. Nevertheless, same techniques can be applied to any other database with similar structure like Yelp and Netflix.
In Section~\ref{sec:data}, we discuss our approach to data labeling and feature extraction. In Section~\ref{sec:approach}, we review our approach to sentiment and category classification of YouTube comments. Results of the experiments for different classification techniques are compared with each other in Section~\ref{sec:exp}. Also, calibration curves are presented and effect of dimensionality reduction on accuracy is examined. Section~\ref{sec:conclusion} concludes the report.
